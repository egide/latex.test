\documentclass[frenchb,pdftex]{scrartcl}
\usepackage[utf8]{inputenc}
\usepackage[T1]{fontenc}
\usepackage[acronym]{glossaries}\makeglossaries
\usepackage{babel,xspace}
\newglossaryentry{printemp}{name={printemp},
	plural={printemps},
	description={Période des amours et des petites fleurs}}
\newglossaryentry{automne}{name={automne},
	plural={automnes},
	description={Période très difficile à supporter à cause de la chaleur intense qu'il y fait}}
\newglossaryentry{ete}{name={été},
	sort={ete},
	plural={étés},
	description={Période très difficile à supporter à cause de la chaleur intense qu'il y fait.
	Avec une phrase très longue qui ne veur rien dire du tout, mais qui semble cependant fort utile
	pour évaluer la manière avec laquelle latex va mettre ceci en forme.\\
	Je voudrais effectivement pouvoir mettre des texte assez long, de vraies définitions, avec, pourquoi pas, des citations dans le texte.
	Cependant, la chose ne me semble pas possible, étant donné que l'on ne peut pas mettre de saut de paragraphe}}
\newacronym[plural={mdrs},
	first={mort de rire (mdr)},
	firstplural={mortde rire (mdr)}]
	{lol}{mdr}{On se bidonne}
\begin{document}

%% Exemple tirée de:
% http://www.developpez.net/forums/d1446384/autres-langages/autres-langages/latex/contribuez/faq-latex/realiser-glossaire-package-glossaries/
%\newglossaryentry{<mot à utiliser pour appeler le terme du glossaire>}
%{%
%	name={<terme>}, % le terme à référencer (l'entrée qui apparaitra dans le glossaire)
%	description={<description>}, % la description du terme (sans retour à la ligne)
%	sort={<terme>}, % si le mot contient des caractère spéciaux, ils ne seront pas pris en compte
%	plural={<termes>} % la forme plurielle du terme
%}
% utilisation: \gls \Gls \glspl \Glspl

C'est vraiment comique, \gls{lol}.
C'est l'\gls{ete}\dots C'est le \Gls{printemp}\dots Les \glspl{automne} sont pleins de couleurs!
J'en suis vraiment \gls{lol}\dots


\printglossaries

\end{document}
