\documentclass[frenchb,pdftex]{scrartcl}
\usepackage[utf8]{inputenc}
\usepackage[T1]{fontenc}
\usepackage{glossaries}
\makeglossaries
\usepackage{babel}
\newglossaryentry{ete}
{%
	name={été},
	description={Période très difficile à supporter à cause de la chaleur intense qu'il y fait.},
	sort={ete},
	plural={étés}
}

\begin{document}

%% Exemple tirée de:
% http://www.developpez.net/forums/d1446384/autres-langages/autres-langages/latex/contribuez/faq-latex/realiser-glossaire-package-glossaries/
%\newglossaryentry{<mot à utiliser pour appeler le terme du glossaire>}
%{%
%	name={<terme>}, % le terme à référencer (l'entrée qui apparaitra dans le glossaire)
%	description={<description>}, % la description du terme (sans retour à la ligne)
%	sort={<terme>}, % si le mot contient des caractère spéciaux, ils ne seront pas pris en compte
%	plural={<termes>} % la forme plurielle du terme
%}

Du texte juste pour voir ce que cela donne avec des accents...


C'est l'\gls{ete}\dots

\printglossary

\end{document}
